\documentclass[11pt]{article}

\usepackage{amsmath}
\usepackage{amsfonts}
\usepackage{enumerate}

\title{CS 344 Assignment 2}
\author{Craig Perkins, Alex Tang, Steve Grzenda}
\date{Due March 14, 2014}

\begin{document}
\maketitle

\section*{Problem 1}
\begin{enumerate}[A.]
    \item
    \item
    \item
    \item
    \item
\end{enumerate}

\section*{Problem 2}
\begin{enumerate}[A.]
    \item
    \item
\end{enumerate}

\section*{Problem 3}
\begin{enumerate}
    \item
    \item
    \item
\end{enumerate}

\section*{Problem 4}

\begin{verbatim}
function getMissingDate():
  boolean puzzles[n] // puzzles is a n-bit array
  for i in [1..n]:
    puzzles[i] = 0

  for i in [1..n]:
    crossword_puzzle puzzle = getPuzzle(i)
    day = getDay(puzzle);
    month = getMonth(puzzle)
    year = getYear(puzzle)
    puzzles[hash(month, day, year)] = 1

  for i in [1..n]:
    if puzzles[i] == 0:
      day = getDay(puzzle)
      month = getMonth(puzzle)
      year = getYear(puzzle)
      return month, day, year
\end{verbatim}

This algorithm will run in $O(n)$ time and take $n$-bits of space, i.e. $O(n)$ 
space. Note that \texttt{hash()} can be any function that maps a 1-to-1 
correspondence between the dates and the indices in the boolean array 
\texttt{puzzles}.

\section*{Problem 5}

\end{document}
