\documentclass[11pt]{article}

\usepackage{amsmath}
\usepackage{amsfonts}
\usepackage{enumerate}
\usepackage{algorithmic}

\title{CS 344 Assignment 5}
\author{Craig Perkins, Alex Tang, Steve Grzenda}
\date{Due May 6, 2014}

\begin{document}
\maketitle

\section*{Problem 1}
\begin{enumerate}[(a)]
    \item 
    \item 
    \item
\end{enumerate}

\section*{Problem 2}
\begin{enumerate}[A.]
    \item 
    \item 
\end{enumerate}

\section*{Problem 3}
\begin{enumerate}[1.]
    \item The problem is that they need to find the maximum number of edges between two nodes by placing m pairs into the two nodes, or "tables". The cost function is $max \sum e(x,y)$ for all x and y
    \item A local search approach would be to start with 1 person in table A and the rest in table B. Move 1 person over to table A and if it improves the cost then keep that person there permanently. Repeat this process until there is no improvement.\\
    This is a 2-approximation polynomial time solution because in the worst case (where everyone hates everyone) the maximum number of edges is m where $ m = \frac{n(n-1)}{2}$. The local search with stop when each side has $\frac{n}{2}$ people so the cost is approximately $\frac{m}{2}$.
    \item A greedy solution would be to find the person with the most amount of hated people. Put that person on table A and all his enemies on table B. Repeat on all unassigned people until none are left.\\
    This is a 2-approximation polynomial time solution for the same reasons listed above 
    \item A randomized approach would be to queue up all the names and alternately assign to table A and B.\\
    This would be a 2-approximation polynomial time solution because with $\frac{n}{2}$ people on either side the number of edges would be about $\frac{m}{2}$
\end{enumerate}

\section*{Problem 4}
\begin{enumerate}[a.]
    \item One way to do this would be to use Karger's algorithm which is a randomized algorithm that contracts edges of a graph. Since it is randomized it does not always work; however since it runs in polynomial time we can run it multiple times and take the min of the min cuts. 
    \item A solution to this problem would be to block all roads in between the cities. Then queue up all the edges. Pop an edge and return it to the graph. Perform 3 DFS's (A to B, A to C, B to C). If a DFS succeeds then that road must be blocked. If it does not succeed then the edge can be returned to the graph. Repeat until the queue is empty.
    \item Once again remove all edges. Add in one edge and perform Floyd Warshall. If the number of infinities equals m-1 then stop. Else save the minimum number of infinities that is $\ge m$. Continue adding in edges until there is no more improvement
    \item
\end{enumerate}

\end{document}
