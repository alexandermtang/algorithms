\documentclass[11pt]{article}

\usepackage{amsmath}

\title{Design \& Analysis of Algorithms HW1}
\author{Craig Perkins, Alex Tang, Steve Grzenda}
\date{}

\begin{document}
\maketitle

\section*{Problem 1}

% Problem 1i
\noindent i.  $f(n) = \sqrt{2^{7x}} \textrm{ , } g(n) = \lg (7^{2x})$

\vspace{3mm}

\indent $f(n) = \sqrt{2^{7x}} = (2^{7x})^{\frac{1}{2}} = 2^{\frac{7}{2}x} = (2^{\frac{7}{2}})^x = a^x \implies f(n)$ is exponential

\indent $g(n) = \lg (7^{2x}) = 2x\lg (7) = (2\lg 7)x = cx \implies g(n)$ is linear

\indent Since f(n) is exponential and g(n) is linear $\implies \boxed{f  = \Omega(g)}$

\vspace{3mm}

% Problem 1ii
\noindent ii. $f(n) = 2^{n\ln n} \textrm{ , } g(n) = n!$

\vspace{3mm}

% Problem 1iii
\noindent iii. $f(n) = \lg (\lg ^{*}n) \textrm{ , } g(n) = \lg ^*(\lg n)$

\vspace{3mm}

% Problem 1iv
\noindent iv. $f(n) = \frac{\lg n^2}{n} \textrm{ , } g(n) = \lg ^*n$

\vspace{3mm}

% Problem 1v
\noindent v. $f(n) = 2^n \textrm{ , } g(n) = n^{\lg n}$

\vspace{3mm}

% Problem 1vi
\noindent vi. $f(n) = 2^{\sqrt{\lg n}} \textrm{ , } g(n) = n(\lg n)^3$

\vspace{3mm}

% Problem 1vii
\noindent vii. $f(n) = e^{\cos n} \textrm{ , } g(n) = \lg n$

\vspace{3mm}

\indent This problem is a bit confusing because you need to consider the range of the $\cos n$ function. Since the $\cos$ function has a range of $-1 \le \cos n \le 1$, that implies that the range of $e^{\cos n}$ is $\frac{1}{e} \le e^{\cos n} \le e$ and that f(n) is actually constant time with respect to the input.

\indent $g(n) = \lg n \implies g(n)$ is logarithmic

\indent Since logarithmic dominates constant time $\implies \boxed{f  = O(g)}$

\vspace{3mm}

% Problem 1viii
\noindent viii. $f(n) =  \lg n^2 \textrm{ , } g(n) = (\lg n)^2$

\vspace{3mm}

% Problem 1ix
\noindent ix. $f(n) = \sqrt{4n^2 - 12n + 9} \textrm{ , } g(n) = n^{\frac{3}{2}}$

\vspace{3mm}

\indent $f(n) = \sqrt{4n^2 - 12n + 9} = \sqrt{(2n-3)^2} = 2n-3 \implies f(n)$ is linear
\indent $g(n) = n^{\frac{3}{2}} \implies g(n)$ is more than linear, but less than quadratic.
\indent Since f(n) and g(n) are both polynomial, but g(n) has a higher power than f(n) $\implies \boxed{f  = O(g)}$

\vspace{3mm}

% Problem 1x
\noindent x. $f(n) = \sum\limits_{k=1}^n k \textrm{ , } g(n) = (n+2)^2$

\vspace{3mm}

\indent $f(n) = \sum\limits_{k=1}^n k = \frac{n(n+1)}{2} = \frac{1}{2}n^2 + \frac{1}{2}n \implies f(n)$ is quadratic

\indent $g(n) = (n+2)^2 = n^2 + 4n + 4 \implies g(n)$ is quadratic

\indent Since f(n) is quadratic and g(n) is also quadratic $\implies \boxed{f  = \Theta(g)}$

\section*{Problem 2}

\section*{Problem 3}

\section*{Problem 4}

\section*{Problem 5}

\section*{Problem 6}

\section*{Problem 7}

\section*{Problem 8}

\section*{Problem 9}

\end{document}
