\documentclass[11pt]{article}

\usepackage{amsmath}
\usepackage{amsfonts}
\usepackage{enumerate}

\title{CS 344 Assignment 1}
\author{Craig Perkins, Alex Tang, Steve Grzenda}
\date{Due February 18, 2014}

\begin{document}
\maketitle

\section*{Problem 1}
\begin{enumerate}

% Problem 1i
\item $f(n) = \sqrt{2^{7n}} \textrm{ , } g(n) = \lg (7^{2n})$ \\ \\
$f(n) = \sqrt{2^{7n}} = (2^{7n})^{\frac{1}{2}} = 2^{\frac{7}{2}n} = (2^{\frac{7}{2}})^n = a^n \implies f(n)$ is exponential \\
$g(n) = \lg (7^{2n}) = 2n\lg (7) = (2\lg 7)n = cn \implies g(n)$ is linear \\
Since f(n) is exponential and g(n) is linear $\implies \boxed{f(n)  = \Omega(g(n))}$ \\

% Problem 1ii
\item $f(n) = 2^{n\ln n} \textrm{ , } g(n) = n!$ \\ \\
 $\lg (f(n)) = \lg 2^{n\ln n} = n\ln n \implies \lg (f(n))$ is linearithmic \\
 $\lg (g(n)) = \lg (n!) \approx n \lg n \implies \lg (g(n))$ is linearithmic. Note that the approximation of $\lg (n!)$ is done using Stirling's approximation. \\
 Since $\lg (f(n))$ and $\lg (g(n))$ are both linearithmic $\implies \boxed{f(n) = \Theta(g(n))}$ \\

% Problem 1iii
\item $f(n) = \lg (\lg ^{*}n) \textrm{ , } g(n) = \lg ^*(\lg n)$ \\ \\
  $f(n) = \lg (\lg ^{*}n) = \lg (1+ \lg (\lg n))$. In the preceding function we can ignore the 1 because it is constant and does not grow with n, so we can say that $f(n) = O(\lg (\lg (\lg n)))$. \\
 $g(n) = \lg ^{*}(\lg n) = 1 + \lg (\lg (\lg (n))$. In the preceding function we can ignore the 1 because it is constant and does not grow with n, so we can say that $g(n) = O(\lg (\lg (\lg n)))$. \\
 Since f(n) and g(n) grow at the same rate as a function of the input $\implies \boxed{f(n) = \Theta(g(n))}$ \\

% Problem 1iv
\item $f(n) = \frac{\lg n^2}{n} \textrm{ , } g(n) = \lg ^*n$ \\ \\
 $f(n) =  \frac{\lg n^2}{n} = \frac{2\lg (n)}{n}$ \\
 Since n dominates $\lg n$ the preceding function will approach 0 as the input approaches $\infty$. (Possibly the only program in the world that gets faster as the input gets larger) \\
 $g(n) = \lg ^{*}(n)$. g(n) is a very slow growing function, but still grows as a function of its input unlike g(n). \\
 Since f(n) takes almost no time to execute for large n and g(n) grows with n, g(n) will dominate f(n) $\implies \boxed{f(n) = O(g(n))}$ \\

% Problem 1v
\item $f(n) = 2^n \textrm{ , } g(n) = n^{\lg n}$ \\ \\
 $f(n) = 2^n \implies \lg (f(n)) = \lg (2^n) = n\cdot \lg (2)$ is linear with respect to the input. \\
 $g(n) = n^{\lg n} \implies \lg (g(n)) = \lg (n^{\lg n}) = (\lg n)^2$ is polylogarithmic with respect to the input. \\
 Since polynomial (linear is a polynomial) functions dominate polylogarithmic functions $\implies \boxed{f(n) = \Omega(g(n))}$  \\

% Problem 1vi
\item $f(n) = 2^{\sqrt{\lg n}} \textrm{ , } g(n) = n(\lg n)^3$ \\ \\
  $\lg(f(n)) = \lg (2^{\sqrt{\lg n}}) = {\sqrt{\lg n}}\cdot \lg 2 = {\sqrt{\lg n}}$ is polylogarithmic with respect to input \\
  $\lg(g(n)) = \lg(n(\lg n)^3) = \lg n + 3\lg (\lg n)$ is logarithmic with respect to input because $\lg (n)$ dominates the expression. \\
 Since $\lg (f(n))$ and $\lg (g(n))$ are both polylogarithmic, but the power of $\lg (g(n))$ is greater than that of $\lg (f(n))$ $\implies \boxed{f(n) = O(g(n))}$ \\

% Problem 1vii
\item $f(n) = e^{\cos n} \textrm{ , } g(n) = \lg n$ \\ \\
 This problem is a bit confusing because you need to consider the range of the $\cos n$ function. Since the $\cos$ function has a range of $-1 \le \cos n \le 1$, that implies that the range of $e^{\cos n}$ is $\frac{1}{e} \le e^{\cos n} \le e$ and that f(n) is actually constant time with respect to the input. \\
 $g(n) = \lg n \implies g(n)$ is logarithmic \\
 Since logarithmic dominates constant time $\implies \boxed{f(n)  = O(g(n))}$ \\

% Problem 1viii
\item $f(n) =  \lg n^2 \textrm{ , } g(n) = (\lg n)^2$ \\ \\
 $f(n) = \lg (n^2) = 2\lg (n)$ is logarithmic with respect to input. \\
 $g(n) = (\lg (n))^2$ is polylogarithmic with respect to input. \\
 Since g(n) is polylogarithmic and the power is greater than 1, g(n) dominates f(n) $\implies \boxed{f(n) = O(g(n))}$ \\

% Problem 1ix
\item $f(n) = \sqrt{4n^2 - 12n + 9} \textrm{ , } g(n) = n^{\frac{3}{2}}$ \\ \\
 $f(n) = \sqrt{4n^2 - 12n + 9} = \sqrt{(2n-3)^2} = 2n-3 \implies f(n)$ is linear \\
 $g(n) = n^{\frac{3}{2}} \implies g(n)$ is more than linear, but less than quadratic. \\
 Since f(n) and g(n) are both polynomial, but g(n) has a higher power than f(n) $\implies \boxed{f(n)  = O(g(n))}$ \\

% Problem 1x
\item $f(n) = \sum\limits_{k=1}^n k \textrm{ , } g(n) = (n+2)^2$ \\ \\
 $f(n) = \sum\limits_{k=1}^n k = \frac{n(n+1)}{2} = \frac{1}{2}n^2 + \frac{1}{2}n \implies f(n)$ is quadratic \\
 $g(n) = (n+2)^2 = n^2 + 4n + 4 \implies g(n)$ is quadratic \\
 Since f(n) is quadratic and g(n) is also quadratic $\implies \boxed{f(n) = \Theta(g(n))}$ \\

\end{enumerate}

\section*{Problem 2}

1. Random Sample: $O(n)$ \\
2. Even Check: $O(1)$ \\
3. \indent Add 1: $O(n)$ \\
4. Mod: $O(n^2)$ \\
5. For loop: $O(n)$ \\
6. \indent GCD $O(n^3)$ \\
7. \indent \indent return False \\
8. \indent Compute x,z $O(n)$ \\
9. \indent Modular Exponentiation $O(n^3)$ \\
10. \indent For loop $O(n)$ \\
11. \indent \indent Modular Exponentiation $O(n^3)$ \\
12. \indent \indent Mod $O(n^2)$ \\
13. \indent Low Error Primality Test $O(n^3)$ \\
14. \indent \indent return FALSE \\
15. return TRUE \\

\noindent \emph{Total:} $O(n^5)$. This was calculated by multiplying the big O times of the outer loop, the inner loop and the dominating modular exponentiation, i.e. $O(n) \cdot O(n) \cdot O(n^3)$.

\section*{Problem 3}

\begin{itemize}

\item Let a tree with a single node have height 0 and define this row as row 0. This means that at row $k$, there will be at most $m^k$ nodes. Therefore, a tree of height $h$ will have at most $N$ nodes where \[N = \sum\limits_{k=0}^h m^k\ = \frac{m^{h+1} - 1}{m-1}.\] Solving for $h$ gives us the lower bound for the height of the tree: \[\boxed{h = \lceil\log_m [N(m-1)+1]\rceil - 1}.\]

\item $h_m = \lceil\log_m [N(m-1)+1]\rceil - 1$ and $h_{m'} = \lceil\log_{m'} [N({m'}-1)+1]\rceil - 1$. $h_m$ and $h_{m'}$ will have the same asymptotic running time, i.e. $h_m = \Theta(h_{m'}) $, because they will both be logarithmic as a function of input and the change of base formula tells us that logarithms of different bases only differ by a constant. The only exception to this rule is if it is a 1-ary tree, in which case the height grows linear and the heights will have different rates of growth. 

\item Using this definition of modular exponentiation, there will be $O(\log y) = O(n)$ recursive calls. Each call involves the multiplication of two $m$-bit numbers, which will take $O(m^2)$ time with the result being at most $2m$ bits. After the multiplication, the $2m$ bit result modulo $N$, which is an $o$-bit number, will take $O(mo + \max(o,m)) = O(mo)$ time. In total, this algorithm will have a running time of $\boxed{O(n \cdot (m^2 + mo))}$.

\end{itemize}

\section*{Problem 4}

\begin{itemize}

\item $2^{902} = (2^3)^{300} \cdot 2^2 \mod 7 \equiv 1^{300} \cdot 2^2 \mod 7 \equiv \boxed{4 \mod 7}$

\item $11^{-1} \mod 120 = 11$ \\
$13^{-1} \mod 45 = 7$ \\
$35^{-1} \mod 77$ does not exist because $gcd(35,77) > 1$ \\
$9^{-1} \mod 11 = 5$ \\
$11^{-1} \mod 1111$ does not exist because $gcd(11,1111) > 1$ \\ \\
To find the modulo multiplicative inverse in the previous problems we used the extended Euclid algorithm. The algorithm can be demonstrated with $9^{-1} \mod 11$ by first finding the $gcd(9,11)$ using the Euclid algorithm:

\[ \textrm{step } 0: 11 = 1(9) + 2\]
\[ \textrm{step } 1: 9 = 4(2) + 1 \]
\[ \textrm{step } 2: 1 = 1(1) + 0 \]

So $gcd(9,11) = 1$. This means that there is a linear combination of 9 and 11 equaling 1. Working backwards using the extended Euclid algorithm:

\[ 1 = 9 - 4(2) \]
\[ 1 = 9-4(11-1(9)) = 5(9) - 4(11)\]

So $9^{-1} \mod 11$ = 5.

\item For a number $x$, if $\forall y \epsilon [1, x-1] : gcd(x,y) = 1$, then x must be prime. The extended Euclid algorithm gives us the multiplicative inverse of $a \mod n$ if and only if $gcd(a, n)=1$. This means that the only numbers in the set $S = \{0, 1, \ldots, x^m - 1\}$ that do not have multiplicative inverses modulo $x^m$ are the numbers in the set \[U = \{a \mid a \in S, gcd(a,x^m)=x\}\] where \[\left\vert{U}\right\vert = \frac{x^m}{x}.\] The $gcd(a, x^m)=x$ because $x$ is prime. Therefore, there are $x^m - \frac{x^m}{x} = x^{m-1}(x-1)$ numbers in $S$ that have multiplicative inverses modulo $x^m$. Since finding the multiplicative inverse using the extended Euclid algorithm takes $O(n^3)$, the total running time takes \[\boxed{O(n^3) \cdot [x^{m-1}(x-1)]}.\]

\end{itemize}

\section*{Problem 5}

\begin{itemize}

  \item True. Let $g = gcd(x, y)$. Then, $x = a_xg$ and $y = a_yg$ such that $a_x$ and $a_y$ do not share any common factors, i.e. $a_x$ and $a_y$ are coprime. Rewrite \[(5x + 3y, 3x + 2y)\] as \[(5a_xg + 3a_yg, 3a_xg + 2a_yg).\] Factoring out $g$ gives \[(g\cdot(5a_x + 3a_y), g\cdot(3a_x + 2a_y)).\] Here, you can see that $(x,y)$ and $(5x + 3y, 3x + 2y)$ have a common divisor, $g$. The greatest common divisor of $(x,y)$ and $(5x + 3y, 3x + 2y)$ cannot be $> g$ because of the following: if $a_y$ shared a common factor with 5, then factor out 5 from $5a_x + 3a_y$. However, you will not be able to factor out 5 from $3a_x + 2a_y$, because 5 and 3 are coprime as are $a_x$ and $a_y$. This same reasoning can be applied to $a_x$. Therefore, the greatest common divisor of $(x,y)$ and $(5x + 3y, 3x + 2y)$ must be $g$.

\item The proof that the numbers of the set S are relatively prime is very similar to that of the Prime Number Theorem. We can show that the first few elements of the set \{2,3,7,42,...\} are coprime and that the n-th element is also coprime because: 

\[s_n \equiv 1 \mod s_0 \]
\[s_n \equiv 1 \mod s_1\]
\[ \vdots \]
\[ s_n \equiv 1 \mod s_{n-1} \]

Since all elements are coprime to those in the set before it there will be no two elements of the set that have a gcd other than 1. 

\end{itemize}

\section*{Problem 6}

\section*{Problem 7}

\begin{itemize}

\item The numbers that are modular multiplicatives of themselves module n are the numbers 1 and $n-1$. 1 is the modulo multiplative inverse of itself because $1\cdot 1 \equiv 1 \mod n$. $n-1$ is the modular multiplicative inverse of itself because $(n-1)(n-1) = n^2-2n+1\equiv 1 \mod n$. 

\item To show that $(n-1)! \equiv -1 \mod n$ and n must be prime, we must first observe that $n-1 \equiv -1 \mod n$. Using this we can see that $(n-1)! \equiv -(n-2)! \mod n$.  From the numbers $2$ to $n-2$ we can also observe that their inverses will also be in the range from $2$ to $n-2$ and be unique. We can be assured that all of 1 to $n-1$ inverse to exist because if a number a is picked from $(1,n-1)$, then gcd(a,n) will be 1 because n is prime and $a < n$. We can also be assured they're unique because assume that a and b are picked from $(1,n-1)$. Then $a\cdot x \equiv 1 \mod n$ and $b\cdot x \equiv 1\mod n$ and $a\cdot x \equiv b \cdot x \mod n$. Dividing the equivalence by x we obtain $a \equiv b \mod n$ and since $1 \le a,b < n$, we can conclude that $a=b$ so that each value is unique (one-to-one mapping). Furthermore the inverse of a number's inverse is itself, meaning that inverses come in pairs (i.e. 2 and 6 $\mod 11$). From that fact we can show that $(n-2)(n-1)\cdots 3\cdot 2 \equiv 1 \mod n$ and from part one we know that $1 \equiv 1 \mod n$. Altogether it can be shown that $(n-1)! \equiv -1 \mod n$.

\item If n is not prime, than the fundamental theorem of arithmetic says that $\exists a \textrm{ such that } 1 < a < n\textrm{ , } gcd(a,n) > 1$. From that fact we can conclude that $(n-1)! \not\equiv -1 \mod n$ because $gcd((n-1)!,n) > 1$

\item One main reason for not implementing this primality test is because it does not scale well with the size of the input. The algorithm might be able to determine with certainty the difference between and prime and not prime, but will take ages to compute. In the study of algorithms the second fastest growing algorithm besides $n^n$ is $n!$, not to mention that a modulus is also being applied. 

\end{itemize}

\section*{Problem 8}

\begin{itemize}

\item To start the proof assume that $\exists (p,q)$ such that $q \equiv m \mod x$, $q \equiv n \mod y$ and $p \equiv m \mod x$, $p \equiv n \mod y$ and $p \ne q$, $0 \le p,q \le xy$, x and y are prime. That means that I can write $p = m + i\cdot x$ and $q = m + j\cdot x$ where $i \ne j$. Substituting in we obtain $m + i\cdot x \equiv n \mod y$ and $m + j\cdot x \equiv n \mod y$. We know from our assumptions that $0 \le m + i\cdot x, m + j\cdot x < xy$. We can also note that $i,j < y$. By subtracting from both sides we get $ix \equiv n - m \mod y$ and $jx \equiv n - m \mod y$. Since $i,j < y$ we can conclude that $i=j$ which contradicts our assumption. This means that $p=q$ and that p is unique to a pair of $(m,n)$. This problem shows the basis of the Chinese Remainder Theorem.

\begin{table}[ht] 
\centering
\caption{Modulus} % title of Table 
\begin{tabular}{c c c} % centered columns (4 columns) 
\hline\hline %inserts double horizontal lines 
n & $\mod 5$ & $\mod 7$ \\ [0.5ex] % inserts table 
%heading 
\hline % inserts single horizontal line 
0 & 0 & 0 \\ % inserting body of the table 
1 & 1 & 1 \\ 
2 & 2 & 2 \\ 
3 & 3 & 3 \\ 
4 & 4 & 4 \\ 
5 & 0 & 5 \\ 
6 & 1 & 6 \\ 
7 & 2 & 0 \\ 
8 & 3 & 1 \\ 
9 & 4 & 2 \\ 
10 & 0 & 3 \\ 
11 & 1 & 4 \\ 
12 & 2 & 5 \\ 
13 & 3 & 6 \\ 
14 & 4 & 0 \\ 
15 & 0 & 1 \\ 
16 & 1 & 2 \\ 
17 & 2 & 3 \\ 
18 & 3 & 4 \\ [1ex] % [1ex] adds vertical space 
\hline %inserts single line 
\end{tabular} 
\quad
\begin{tabular}{c c c}
\hline\hline %inserts double horizontal lines 
n & $\mod 5$ & $\mod 7$ \\ [0.5ex] % inserts table 
%heading 
\hline % inserts single horizontal line 
19 & 4 & 5 \\ 
20 & 0 & 6 \\ 
21 & 1 & 0 \\ 
22 & 2 & 1 \\ 
23 & 3 & 2 \\ 
24 & 4 & 3 \\ 
25 & 0 & 4 \\ 
26 & 1 & 5 \\ 
27 & 2 & 6 \\ 
28 & 3 & 0 \\ 
29 & 4 & 1 \\
30 & 0 & 2 \\
31 & 1 & 3 \\
32 & 2 & 4 \\
33 & 3 & 5 \\
34 & 4 & 6 \\
35 & 0 & 0 \\ 
36 & 1 & 1 \\ 
37 & 2 & 2 \\ [1ex] % [1ex] adds vertical space 
\hline
\end{tabular} 
\label{table:nonlin} % is used to refer this table in the text 
\end{table} 

\item This is the Chinese Remainder Theorem for $n = 2$. Let 
  \[q = (m\cdot c_x + n\cdot c_y) \mod xy\] 
where $c_x= y(y^{-1} \mod x)$ and $c_y= x(x^{-1} \mod y)$.
Take $q \pmod x$. Then 
  \[q = (m\cdot c_x + n\cdot c_y) \mod xy \pmod x\] 
Because $c_y\pmod x = 0$,
  \[q = mc_x \pmod x\]
Because $c_x\pmod x = 1$,
  \[q = m \pmod x\]
Taking $q \pmod y$, you get $q = n \pmod y$ by the same reasoning. So $q$ is a solution to both $q=m \pmod x$ and $q=n \pmod y$.

\item This will be the case of the Chinese Remainder Theorem for $n=3$. The above equations and properties will still hold for 3 primes. That is that $c_x \equiv 1 \mod x$, $c_x \equiv 0 \mod y$ and $c_x \equiv 0 \mod z$. The corresponding $q$ can be computed as follows and is useful for number 9.
\[q = m\cdot(yz\cdot((yz)^{-1} \mod x)) + n\cdot(xz\cdot((xz)^{-1} \mod y))+o\cdot(xy\cdot((xy)^{-1} \mod z))\mod xyz\] 

\end{itemize}

\section*{Problem 9}

\noindent Mallory was able to figure out this message by using the Chinese Remainder Theorem. This theorem can be applied when you have a $q$ that you wish to find that is equivalent to an x, y, and z on the condition that they are all relative primes. This applies in this case because every e in the public keys is 3 which opens a vulnerability in RSA\\
i.e :\\
$q \equiv m \mod x$ \\
$q \equiv n \mod y$ \\
$q \equiv o \mod z$ \\

\noindent Then to solve for q the formula is:\\
\indent $q = m \cdot c_x + n \cdot c_y + o \cdot c_z$\\
where $c_x$, $c_y$, and $c_z$ are: \\
\indent$c_x = y \cdot z \cdot (y^{-1} \mod x)(z^{-1} \mod x)$\\
\indent$c_y = x \cdot z \cdot (x^{-1} \mod y)(z^{-1} \mod y)$\\
\indent$c_z = y \cdot x \cdot (y^{-1} \mod z)(x^{-1} \mod z)$\\

\noindent In Mallory's case $q = k^3$ (where $k$ is the message) and the equations she used are as follows: \\

$q \equiv 674 \mod 3337$ \\
\indent$q \equiv 36 \mod 187$ \\
\indent$q \equiv 948 \mod 1219$ \\

$q = 674 \cdot c_x + 36 \cdot c_y + 948 \cdot c_z$\\

$c_x = 187 \cdot 1219 \cdot (187^{-1} \mod 3337)(1219^{-1} \mod 3337)$\\
\indent$c_y = 3337 \cdot 1219 \cdot (3337^{-1} \mod 187)(1219^{-1} \mod 187)$\\
\indent$c_z = 187 \cdot 3337 \cdot (187^{-1} \mod 1219)(3337^{-1} \mod 1219)$\\

\noindent After plugging in the values for $c_x$, $c_y$, and $c_z$, she got $q = 74088$ and $k = \sqrt[3]{q}$ and therefore $k = 42$.

\end{document}
