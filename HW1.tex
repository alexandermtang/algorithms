\documentclass[11pt]{article}

\usepackage{amsmath}
\usepackage{amsfonts}
\usepackage{enumerate}

\title{CS 344 Assignment 1}
\author{Craig Perkins, Alex Tang, Steve Grzenda}
\date{Due February 18, 2014}

\begin{document}
\maketitle

\section*{Problem 1}
\begin{enumerate}

% Problem 1i
\item $f(n) = \sqrt{2^{7n}} \textrm{ , } g(n) = \lg (7^{2n})$ \\ \\
$f(n) = \sqrt{2^{7n}} = (2^{7n})^{\frac{1}{2}} = 2^{\frac{7}{2}n} = (2^{\frac{7}{2}})^n = a^n \implies f(n)$ is exponential \\
$g(n) = \lg (7^{2n}) = 2n\lg (7) = (2\lg 7)n = cn \implies g(n)$ is linear \\
Since f(n) is exponential and g(n) is linear $\implies \boxed{f  = \Omega(g)}$ \\

% Problem 1ii
\item $f(n) = 2^{n\ln n} \textrm{ , } g(n) = n!$ \\ \\
 $\lg (f(n)) = \lg 2^{n\ln n} = n\ln n \implies \lg (f(n))$ is linearithmic \\
 $\lg (g(n)) = \lg (n!) \approx n \lg n \implies \lg (g(n))$ is linearithmic. Note that the approximation of $\lg (n!)$ is done using Stirling's approximation. \\
 Since $\lg (f(n))$ and $\lg (g(n))$ are both linearithmic $\implies \boxed{f(n) = \Theta(g(n))}$ \\

% Problem 1iii
\item $f(n) = \lg (\lg ^{*}n) \textrm{ , } g(n) = \lg ^*(\lg n)$ \\ \\
 $f(n) = \lg (\lg ^{*}n = \lg (1+ \lg (\lg (n))$. In the preceding function we can ignore the 1 because it is constant and does not grow with n, so we can say that $f(n) = O(\lg (\lg (\lg (n))))$ \\
 $g(n) = \lg ^{*}(\lg (n) = 1 + \lg (\lg (\lg (n))$. In the preceding function we can ignore the 1 because it is constant and does not grow with n, so we can say that $g(n) = O(\lg (\lg (\lg (n))))$ \\
 Since f(n) and g(n) grow at the same rate as a function of the input $\implies \boxed{f(n) = \Theta(g(n))}$ \\

% Problem 1iv
\item $f(n) = \frac{\lg n^2}{n} \textrm{ , } g(n) = \lg ^*n$ \\ \\
 $f(n) =  \frac{\lg n^2}{n} = \frac{2\lg (n)}{n}$ \\
 Since n dominates $\lg n$ the preceding function will approach 0 as the input approaches $\infty$. (Possibly the only program in the world that gets faster as the input gets larger) \\
 $g(n) = \lg ^{*}(n)$. g(n) is a very slow growing function, but still grows as a function of its input unlike g(n). \\
 Since f(n) takes almost no time to execute for large n and g(n) grows with n, g(n) will dominate f(n) $\implies \boxed{f(n) = O(g(n))}$ \\

% Problem 1v
\item $f(n) = 2^n \textrm{ , } g(n) = n^{\lg n}$ \\ \\
 $f(n) = 2^n \implies \lg (f(n)) = \lg (2^n) = n\cdot \lg (2)$ is linear with respect to the input. \\
 $g(n) = n^{\lg n} \implies \lg (g(n)) = \lg (n^{\lg n}) = (\lg (n))^2$ is polylogarithmic with respect to the input. \\
 Since polynomial (linear is a polynomial) functions dominate polylogarithmic functions $\implies \boxed{f(n) = \Omega(g(n))}$  \\

% Problem 1vi
\item $f(n) = 2^{\sqrt{\lg n}} \textrm{ , } g(n) = n(\lg n)^3$ \\ \\
 $f(n) = 2^{\sqrt{\lg n}} \implies \lg (2^{\sqrt{\lg n}}) = {\sqrt{\lg n}}\cdot \lg 2$ is polylogarithmic with respect to input \\
 $g(n) = n(\lg (n))^3 \implies \lg (g(n)) = \lg (n) + 3\lg (\lg (n))$ is logarithmic with respect to input because $\lg (n)$ dominates the expression. \\
 Since $\lg (f(n))$ and $\lg (g(n))$ are both polylogarithmic, but the power of $\lg (g(n))$ is greater than that of $\lg (f(n))$ $\implies \boxed{f(n) = O(g(n))}$ \\

% Problem 1vii
\item $f(n) = e^{\cos n} \textrm{ , } g(n) = \lg n$ \\ \\
 This problem is a bit confusing because you need to consider the range of the $\cos n$ function. Since the $\cos$ function has a range of $-1 \le \cos n \le 1$, that implies that the range of $e^{\cos n}$ is $\frac{1}{e} \le e^{\cos n} \le e$ and that f(n) is actually constant time with respect to the input. \\
 $g(n) = \lg n \implies g(n)$ is logarithmic \\
 Since logarithmic dominates constant time $\implies \boxed{f  = O(g)}$ \\

% Problem 1viii
\item $f(n) =  \lg n^2 \textrm{ , } g(n) = (\lg n)^2$ \\ \\
 $f(n) = \lg (n^2) = 2\lg (n)$ is logarithmic with respect to input. \\
 $g(n) = (\lg (n))^2$ is polylogarithmic with respect to input. \\
 Since g(n) is polylogarithmic and the power is greater than 1, g(n) dominates f(n) $\implies \boxed{f(n) = O(g(n))}$ \\

% Problem 1ix
\item $f(n) = \sqrt{4n^2 - 12n + 9} \textrm{ , } g(n) = n^{\frac{3}{2}}$ \\ \\
 $f(n) = \sqrt{4n^2 - 12n + 9} = \sqrt{(2n-3)^2} = 2n-3 \implies f(n)$ is linear \\
 $g(n) = n^{\frac{3}{2}} \implies g(n)$ is more than linear, but less than quadratic. \\
 Since f(n) and g(n) are both polynomial, but g(n) has a higher power than f(n) $\implies \boxed{f  = O(g)}$ \\

% Problem 1x
\item $f(n) = \sum\limits_{k=1}^n k \textrm{ , } g(n) = (n+2)^2$ \\ \\
 $f(n) = \sum\limits_{k=1}^n k = \frac{n(n+1)}{2} = \frac{1}{2}n^2 + \frac{1}{2}n \implies f(n)$ is quadratic \\
 $g(n) = (n+2)^2 = n^2 + 4n + 4 \implies g(n)$ is quadratic \\
 Since f(n) is quadratic and g(n) is also quadratic $\implies \boxed{f  = \Theta(g)}$ \\

\end{enumerate}

\section*{Problem 2}

\section*{Problem 3}

\vspace{3mm}

\noindent i. Let a tree with a single node has height 0. A lower bound for the height is obtained when a row of nodes gets filled before the next level starts to get populated. A tree with the h-th row filled will have $N = m^h+1$ nodes if $h > 0$. This gives us a lower bound of $\boxed{h = \lceil \log _{m}(N-1)\rceil }$

\vspace{3mm}

\noindent ii. Both $h_m$ and $h_{m'}$ will have the same running time (Big Theta), because they will both be logarithmic as a function of input and the change of base formula tells us that logarithms of different bases only differ by a constant. The only exception to this rule is if it is a 1-ary tree, in which case the height grows linear and the heights will have different rates of growth.

\vspace{3mm}

\noindent iii.

\section*{Problem 4}

\noindent i. $2^{902} = (2^3)^{300} \cdot 2^2 \mod 7 \equiv 1^{300} \cdot 2^2 \mod 7 \equiv $
\framebox{$4 \mod 7$}

\vspace{3mm}

\noindent ii. $11^{-1} \equiv x \mod 120 \implies 11x \equiv 1 \mod 120 \implies x = 11 + (120 \cdot z) \textrm{ where } x \in \mathbb{Z}$ 

\section*{Problem 5}

\vspace{3mm}

\noindent i. True. Let $g = gcd(x, y)$. $x = a_xg y = a_yg$




\section*{Problem 6}

\section*{Problem 7}

\section*{Problem 8}

\section*{Problem 9}

\end{document}
